\begin{abstract}
In recent years the energy-efficiency of software has become a key focus for both researchers and
software developers, aiming to reduce greenhouse-gas emissions and operational costs. Despite this
growing awareness, developers still lack effective strategies to improve the energy-efficiency of
their programs beyond the well-established approaches that optimize for runtime performance. In this
paper we present a dynamic adaptation algorithm that uses energy consumption feedback to optimize
the energy-efficiency of data-parallel applications, by steering the level of parallelism during
runtime through external control. This approach is especially suited to functional languages, whose
side-effect-free nature and strong semantic guarantees allow for easier code generation and
straightforward scalability of the parallelism of programs.

Through a series of experiments we evaluate the effectiveness of our approach. We measure how well
the adaptation algorithm adapts to runtime changes, and we evaluate its effectiveness compared to a
hypothesized oracle that knows the optimal level of parallelism, as well as a
runtime-optimising-based approach. We show that in a fixed-workload scenario we approach the
theoretical best energy-efficiency, and that in changing workload scenarios the adaptation algorithm
converges towards an optimal level of parallelism that minimizes
energy consumption.

\keywords{Dynamic Adaptation, Runtime Systems, Energy-Efficiency, Sustainability,
High-Performance Computing, Parallel Programming}
\end{abstract}
